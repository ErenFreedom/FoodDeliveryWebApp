\documentclass{article}

% Recommended packages for language encoding, fonts, and layout
\usepackage[T1]{fontenc}
\usepackage[utf8]{inputenc}
\usepackage{lmodern}
\usepackage{microtype}
\usepackage{geometry}
\usepackage{graphicx}
\usepackage{hyperref}

% Set the margins
\geometry{a4paper, margin=1in}

% Document metadata
\title{Software Requirements Specification (SRS)}
\author{Your Team Name}
\date{\today}

\begin{document}

% Generates the title
\maketitle

% Table of Contents
\tableofcontents
\newpage

% Introduction section
\section{Introduction}
\subsection{Purpose}
% The purpose of this SRS and its intended audience.

\subsection{Document Conventions}
% Any standards or typographical conventions followed.

\subsection{Intended Audience and Reading Suggestions}
% Different types of readers that the document is intended for.

\subsection{Project Scope}
% A short description of the software being specified and its target users.

\subsection{References}
% A complete list of all documents referenced elsewhere in the SRS.

% Overall Description section
\newpage
\section{Overall Description}
\subsection{Product Perspective}
% The context and origin of the product being specified.

\subsection{Product Functions}
% The major functions the product must perform or must let the user perform.

\subsection{User Classes and Characteristics}
% The different types of users who will use this product.

\subsection{Operating Environment}
% The environment in which the software will operate.

\subsection{Design and Implementation Constraints}
% Any items or issues that will limit the options available to the developers.

\subsection{User Documentation}
% The user documentation components that will be provided.

\subsection{Assumptions and Dependencies}
% Any assumed factors that could affect the requirements stated in the SRS.

% External Interface Requirements section
\newpage
\section{External Interface Requirements}
\subsection{User Interfaces}
% The logical characteristics of each interface between the software product and its users.

\subsection{Hardware Interfaces}
% The logical and physical characteristics of each interface between the software and the hardware components of the system.

\subsection{Software Interfaces}
% The connections between this software and other specific software components.

\subsection{Communications Interfaces}
% The requirements associated with any communications functions required by this product.

% System Features section
\newpage
\section{System Features}
\subsection{System Feature 1}
% A critical system feature.

\subsection{System Feature 2}
% Another critical system feature.

% Repeat subsections as necessary for additional system features.

% Other Nonfunctional Requirements section
\newpage
\section{Nonfunctional Requirements}
\subsection{Performance Requirements}
% The quantitative attributes of the system such as response time, throughput, etc.

\subsection{Safety Requirements}
% The requirements concerned with possible loss, damage, or harm from using the product.

\subsection{Security Requirements}
% The requirements to protect the software from unintended or unauthorized access.

\subsection{Software Quality Attributes}
% Additional quality characteristics that the product must have.

% Updation section
\newpage
\section{Updation}
\subsection{Software Update Requirements}
% The requirements for software updates, including procedures and frequency.

\subsection{Data Migration and Updates}
% The requirements for data migration, including strategies and timing.

\subsection{Document Update Process}
% The process for updating this SRS document, including version control and update frequency.

\end{document}
